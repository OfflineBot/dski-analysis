
\subsection{Ableitungsregeln}
\label{sec:ableitungs_regeln}
Aufbau: Name der Regel. \\
Erste Formel die allgemeine Formel (neutral). \\
Zweite Formel (optional) ein Zahlenbeispiel.


\textbf{Konstantenregel}: 
\[
\begin{aligned}
    f(x) &= n \\
    \implies f'(x) &= 0
\end{aligned}
\]

\textbf{Potenzregel}:
\[
\begin{aligned}
    f(x) &= x^n \\
    \implies f'(x) &= n\cdot x^{n - 1}
\end{aligned}
\]

\textbf{Faktorregel}:
\[
\begin{aligned}
    f(x) &= a\cdot g(x) \\
    \implies f'(x) &= a\cdot g'(x)
\end{aligned}
\]


\textbf{Summenregel/Differenzregel}:
\[
\begin{aligned}
    f(x) &= g(x) + h(x) \\
    \implies f'(x) &= g'(x) + h'(x) 
\end{aligned}
\]

\textbf{Produktregel}: 
\[
\begin{aligned}
    f(x) &= u(x) \cdot v(x) \\
    \implies f'(x) &= u'(x) \cdot v(x) + u(x) \cdot v'(x)
\end{aligned}
\]

\textbf{Kettenregel}:
\[
\begin{aligned}
    f(x) &= u(v(x)) \\
    \implies f'(x) &= u'(v(x)) \cdot v'(x)
\end{aligned}
\]


\subsubsection{Zu Beachten}
\begin{itemize}
    \item $ln(x)$ abgeleitet ist $\frac{1}{x}$
    \item $e^x$ abgeleitet bleibt gleich ($e^x$). Bei $e$ wird die Kettenregel angewendet.
\end{itemize}



\subsection{Monotonie und Grümmung}
\textbf{Montonie}: \\
Monotonie bezieht sich auf das Verhalten einer Funktion in Bezug darauf, 
ob sie stetig zunimmt oder abnimmt. 
Eine Funktion wird als monoton steigend bezeichnet, 
wenn für zwei Punkte $x_1$ und $x_2$ mit $x_1 < x_2$ der Funktionswert an $x_1$ kleiner oder gleich dem Funktionswert an $x_2$ ist. 
Umgekehrt wird eine Funktion als monoton fallend bezeichnet, 
wenn für zwei Punkte $x_1$ und $x_2$ mit $x_1>x_2$ der Funktionswert an $x_1$ größer oder gleich dem Funktionswert an $x_2$ ist. \\\\
\
\textbf{Krümmung}: \\
Die Krümmung einer Funktion beschreibt, wie stark eine Kurve von einer Geraden abweicht. 
Mathematisch gesehen wird die Krümmung einer Funktion durch die zweite Ableitung der Funktion beschrieben. 
Eine positive zweite Ableitung bedeutet, 
dass die Funktion eine nach oben geöffnete Krümmung (konkav) hat, 
während eine negative zweite Ableitung eine nach unten geöffnete Krümmung (konvex) anzeigt. 
Eine Krümmung von null bedeutet, dass die Funktion an dieser Stelle eine Wendepunkt hat, 
wo die Krümmung ihre Richtung ändert.


\subsection{Extrem- und Wendepunkte}
\label{sec:extremundwendepunkte}
\textbf{Extrempunkte}: \\
Extrempunkte sind Punkte auf dem Graphen einer Funktion, 
an denen die Funktion entweder ein lokales Maximum oder Minimum erreicht.
\begin{itemize}
    \item \textbf{Lokales Maximum}: 
        An einem lokalen Maximum ist der Funktionswert größer als in der unmittelbaren Umgebung. Mathematisch bedeutet dies, 
        dass die erste Ableitung der Funktion an diesem Punkt null ist ($f'(x)=0$) und die zweite Ableitung negativ ist ($f''(x)<0$).
    \item \textbf{Lokales Minimum}: 
        An einem lokalen Minimum ist der Funktionswert kleiner als in der unmittelbaren Umgebung. 
        Hier ist ebenfalls die erste Ableitung null ($f'(x)=0$), 
        jedoch ist die zweite Ableitung positiv ($f''(x)>0$).
\end{itemize} 
\
\\
\textbf{Wendepunkte}: \\
Ein Wendepunkt ist ein Punkt auf dem Graphen einer Funktion, 
an dem sich das Krümmungsverhalten ändert. 
Das bedeutet, die Funktion wechselt an diesem Punkt von konkarv (rechtsgekrümmt) zu konvex (linksgekrümmt) oder umgekehrt.
\begin{itemize}
    \item \textbf{Bestimmung von Wendepunkten}: 
        Ein Wendepunkt liegt vor, 
        wenn die zweite Ableitung der Funktion null ist ($f''(x)=0$) und die dritte Ableitung nicht null ist ($f'''(x)\neq 0$). 
    \item Gilt für einen Wendepunkt $x_0$ zusätzlich $f'(x_0) = 0$, so handelt es sich um einen Sattelpunk
\end{itemize}


\subsection{Partielle Differentiation}

Differentiation von Funktionen mit mehreren unabhängigen Variablen.

\textbf{Beispiel}: Der Ertrag (Output) hängt von mehreren Inputfaktoren (Faktor A und Faktor B) ab.
Oft kann nur einer der Inputfaktoren erhöht werden, während der andere Faktor konstant bleibt.

\textbf{Allgemein}: 
\[
    z = f(x,y)
\]
Für die Steigung in Richtung der x/y-Achse gilt, dass der jeweils anderer Wert konstant ist.

Da die jeweils unabhängige Variable konstant gehalten wird, entspricht die Vorgehensweise der partiellen Differentiation der Vorgehensweise der Differentiation mit einer Variable.

Ist eine reelle Funktion $z = f(x,y)$ partiell differenzierbar nach beiden Variablen, dann heißt der Vektor: 

TODO!(HIER FEHLT NOCH RICHTIGES RECHENBEISPIEL)!


\subsection{Extremwerte bei Funktionen mit mehrere Variablen}

