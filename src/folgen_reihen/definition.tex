
\subsection{Definition}
Eine Funktion, die jeder natürlichen Zahl $n\epsilon\mathbb{N}$ eine reelle Zahl zuordnet, heißt: Zahlenfolge und wird mit $a_1, a_2, a_3, ... a_n, ...$ oder $\{ a_n \}$ mit $n\epsilon\mathbb{N}$ bezeichnet. Die $a_n$ heißen glieder der Folge; $a_1$ heißt Anfangsglied.

$1, 2, 3, 4, 5, ...$ | $a_n = n$

$1, \frac{1}{2}, \frac{1}{3}, \frac{1}{4}, \frac{1}{5}, ...$ | $a_n = \frac{1}{n}$

$\frac{1}{2}, \frac{2}{3}, \frac{3}{4}, \frac{4}{5}, \frac{5}{6}, ...$ | $a_n = \frac{n}{n+1}$

$2, 4, 8, 16, 32, 64, ...$ | $a_n = 2^n$


\subsubsection{Arithmetische Folgen}

Eine Folge $\{ a_n\}$, bei der für alle $n\epsilon\mathbb{N}$ gilt:

$a_{n+1} - a_n = d = const.$, heißt \textbf{arithmetische Folge}. $d$ Steht für die differenz.

\point{Beispiel}

$5, 8, 11, 14, 17$ | $d = 3$

$11, 6, 1, -4, -9$ | $d = -5$

Wegen der konstanten Differenz $d$ zweier Folgenglieder ist eine arithmetische Folge eindeutig durch das Anfangsglied $a_1$ und die konstante differenz $d$ bestimmt.

$a_1; a_2; a_3; ... ; a_n$

$a_1; a_1 + d; a_1 + 2d; ... ; a_1 + (n-1)d$

\point{Allgemeines Folgenglied einer arithmetischen Folge} $a_n = a_1 + (n - 1)d$


\subsubsection{Geometrische Folgen}

Eine Folge $\{ a_n\}$, bei für alle $n\epsilon\mathbb{N}$ gilt: 

$\frac{a_{n+1}}{a_n} = q = const.$ heißt \textbf{geometrische Folge}. $q$ steht für den Quotient.

\point{Beispiel}

$2, 4, 8, 16, 32, ...$ | $q = 2$

$\frac{4}{3}, \frac{4}{9}, \frac{4}{27}, \frac{4}{81}, ...$ | $q = \frac{1}{3}$

Wegen des konstanten Quotienten $q$ ist eine geometrische Folge eindeutig durch $a_1$ und $q$ bestimmt.

$a_1, a_2, a_3, a_4, ..., a_n$

$a_1, a_1 \cdot q, a_1 \cdot q^2, a_1 \cdot q^3, ... , a_1 \cdot q^n$

