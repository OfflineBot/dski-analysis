
\subsection{Folgen - Definition}
Eine Funktion, die jeder natürlichen Zahl $n\epsilon\mathbb{N}$ eine reelle Zahl zuordnet, heißt: Zahlenfolge und wird mit $a_1, a_2, a_3, ... a_n, ...$ oder $\{ a_n \}$ mit $n\epsilon\mathbb{N}$ bezeichnet. Die $a_n$ heißen glieder der Folge; $a_1$ heißt Anfangsglied.

$1, 2, 3, 4, 5, ...$ | $a_n = n$ | arithmetisch

$1, \frac{1}{2}, \frac{1}{3}, \frac{1}{4}, \frac{1}{5}, ...$ | $a_n = \frac{1}{n}$ | nicht arithm., nicht geom.

$\frac{1}{2}, \frac{2}{3}, \frac{3}{4}, \frac{4}{5}, \frac{5}{6}, ...$ | $a_n = \frac{n}{n+1}$ | nicht arithm. nicht geom.

$2, 4, 8, 16, 32, 64, ...$ | $a_n = 2^n$ | geometrische Folge


\subsubsection{Arithmetische Folgen}

Eine Folge $\{ a_n\}$, bei der für alle $n\epsilon\mathbb{N}$ gilt:

$a_{n+1} - a_n = d = const.$, heißt \textbf{arithmetische Folge}. $d$ Steht für die differenz.

\point{Beispiel}

$5, 8, 11, 14, 17$ | $d = 3$

$11, 6, 1, -4, -9$ | $d = -5$

Wegen der konstanten Differenz $d$ zweier Folgenglieder ist eine arithmetische Folge eindeutig durch das Anfangsglied $a_1$ und die konstante differenz $d$ bestimmt.

$a_1; a_2; a_3; ... ; a_n$

$a_1; a_1 + d; a_1 + 2d; ... ; a_1 + (n-1)d$

\point{Allgemeines Folgenglied einer arithmetischen Folge} $a_n = a_1 + (n - 1)d$


\subsubsection{Geometrische Folgen}

Eine Folge $\{ a_n\}$, bei für alle $n\epsilon\mathbb{N}$ gilt: 

$\frac{a_{n+1}}{a_n} = q = const.$ heißt \textbf{geometrische Folge}. $q$ steht für den Quotient.

\point{Beispiel}

$2, 4, 8, 16, 32, ...$ | $q = 2$

$\frac{4}{3}, \frac{4}{9}, \frac{4}{27}, \frac{4}{81}, ...$ | $q = \frac{1}{3}$

Wegen des konstanten Quotienten $q$ ist eine geometrische Folge eindeutig durch $a_1$ und $q$ bestimmt.

$a_1, a_2, a_3, a_4, ..., a_n$

$a_1, a_1 \cdot q, a_1 \cdot q^2, a_1 \cdot q^3, ... , a_1 \cdot q^{n-1}$

\point{Allgemeines Folgenglied einer geometrischen Folge} $a_n = a_1 \cdot q^{n-1}$


\subsection{Reihen - Definition}
Gegeben sei eine Zahlenfolge $a_1 + a_2 + a_3 + ... $: man bezeichnet die Addition der Folgenglieder als \textbf{unendliche Reihe} oder \textbf{Reihe}. 

Die Summe $\displaystyle S_n = \sum_{i=1}^{n} a; = a_1 + a_2 + ... + a_n$ heißt $n$-te Partialsumme oder $n$-te Teilsumme. 

Eine arithmetische/geometrische Reihe ist eine Reihe, deren Glieder den Gesetzen einer arithmetischen/geometrischen Folge gehorchen.

\subsubsection{n-te Partialsumme einer arithmetischen Reihe}
Die $n$-te Partialsumme $S_n$ einer arithmetischen Folge:

1. $S_n = a_1 + a_1 + d + ... + a_1 + (n - 2)d + a_1 + (n - 1)d$

2. $S_n = a_1 + (n - 1)d + a_1 + (n - 2)d + ... + a_1 + d + a_1$

1. + 2. $2S_n = [2a_1 + (n - 1)d] + [2a_1 + (n - 1)d] + ... + [2a_1 + (n - 1)d] + [2a_1 + (n - 1)d]$ 

$\Rightarrow n$-mal

$2S_n = n \cdot (2a_1 + (n - 1)d) \ | \ : \ 2$

\point{n-te Partialsumme einer arithmetischen Reihe} $S_n = \frac{n}{2}(2a_1 + (n-1)d)$

$S_n = \frac{n}{2}(2a_1 + (n-1)d)$

$S_n = \frac{n}{2}(a_1 + a_1 + (n - 1)d)$

$S_n = \frac{n}{2}(a_1 + a_n)$


\subsubsection{Gauss'sche Aufgabe}

\[
    S_{100} = \sum_{i=1}^{100} i = 1 + 2 + 3 + 4 + ... + 100 =
\]

$\frac{100}{2} \cdot (1 + 100) = \frac{100 \cdot 101}{2} = 5050$

Summe der natürlichen Zahlen $1, 2, 3, 4, 5, ..., n$

$S_n = \frac{n\cdot (n+1)}{2}$

\subsubsection{n-te Partialsumme einer geometrischen Reihe}
Die $n$-te Partialsumme $S_n$ einer geometirschen Folge:

1. $S_n = a_1 + a_1 \cdot q + a_1 \cdot q^2 + ... + a_1 \cdot q^{a-2} + a_1 \cdot q^{n-1}$

2. $S_n = a_1 \cdot q^{n-1} + a_1 \cdot q^{a-2} + ... + a_1 \cdot q^2 + a_1 \cdot q + a_1$

1. + 2. $q \cdot S_n - S_n = -a_1 + a_1 q^n$ 

$q \cdot S_n - S_n = \frac{a_1 \cdot (-1 + q^n)}{-a_1 + a_1 \cdot q}$ 

\point{n-te Partialsumme einer geometrischen Reihe} $S_n \cdot (q - 1) = a_1 \cdot \frac{q^n - 1}{q - 1}$


\subsection{Folgen - Eigenschaften}

\subsubsection{Beschränkung}
Eine Folge $\{a_n\}$ heißt beschränkt, wenn für alle glieder der Folge gilt: 

$|a_n| \leq c = const.$, $c$ heißt Schranke

\point{Beispiel}

$a_n = (-2)^n$ ist nicht beschränkt, da die Folgenglieder beliebig groß bzw. klein werden können.

\point{Beispiel 2}

$a_n = \frac{1}{n}$ ist beschränkt, da für alle Folgendglieder gilt: $|a_n| \leq 1 \Rightarrow$ Schranke $c = 1$

$0 < \frac{1}{u} \leq 1$


\subsubsection{Monotonie}
Gilt für eine Zahlenfolge $\{a_n\}$:

$a_n < / \leq / = / \geq / > a_{n+1}$ für alle $n\epsilon\mathbb{N}$, so heißt die Folge: 
\begin{itemize}
    \item streng monoton wachsend
    \item monoton wachsend
    \item monoton fallend
    \item streng monoton fallend
\end{itemize}

\point{Beispiel}

$a_n = 4n-3$ 

$4(n + 1)-3 = a_{n+1}$ 

$a_n < a_{n+1} \Rightarrow 4n$ unterschied $\Rightarrow$ streng monton steigend

\point{Beispiel 2}

$a_n = n - n^2$

$(n + 1) - (n + 1)^2 = a_{n+1}$

$n + 1 - (n^2 + 2n + 1) = a_{n+1}$

$n - n^2 - 2n = a_{n+1}$

$a_n = n - n^2 > n-n^2 - 2n = a_n + 1$ | $n - n^2 = a_n$

$a_n$ ist streng monoton fallend

\point{Beispiel 3}

$a_n = 2$ ist monoton wachsend und monoton fallend ($2, 2, 2, 2, ...$)

\point{Beispiel 4}

$a_n = \frac{n}{2^n}$  | $\frac{1}{2}, \frac{2}{4}, \frac{3}{8}, ...$

ist monton fallend da $\frac{1}{2} = \frac{2}{4}$


\subsubsection{Grenzwerte}

Die Glieder der Folge $a_n = 2n (2, 4, 6, 8, 10, ...)$  werden mit wachsendem $n$ immer größer. Man sagt: "Die Folge wächst über alle Grenzen."

Gleiches gilt für $a_n = -3n (-3, -6, -9, -12, ...)$.

Die Folge $a_n = \frac{1}{n} (1, \frac{1}{2}, \frac{1}{3}, \frac{1}{4}, ...)$ nähert sich mit größer werdenen $n$ dem Wert $0$ an. Der Wert $0$ wird nicht erriecht.

Man schreibt $\lim_{n\rightarrow \inf}\frac{1}{n} = 0$

"Limes $\frac{1}{n}$ für n geht gegen $\inf$ ist gleich $0$"

Eine Folge, deren Glieder einem Grenzwert zustreben, heißt \textbf{konvergente Folge} (gegenstück ist die \textbf{divergente Folge}).

Eine Folge mit dem grenzwert 0 heißt \textbf{Nullfolge}.

\point{Beispiel}

$a_n = (-1)^n + \frac{1}{n}$

$a_1 = (-1)^1 + \frac{1}{1} = \ \ 0$

$a_2 = (-1)^2 + \frac{1}{2} = \ \ \frac{3}{2}$

$a_3 = (-1)^3 + \frac{1}{3} = -\frac{2}{3}$

$a_4 = (-1)^4 + \frac{1}{4} = \ \ \frac{5}{4}$

$a_4 = (-1)^5 + \frac{1}{5} = -\frac{4}{5}$

$a_4 = (-1)^6 + \frac{1}{6} = \ \ \frac{7}{6}$

$a_4 = (-1)^7 + \frac{1}{7} = -\frac{6}{7}$

$(-1)^n = > 0 \text{ für n gerade;} < 0 \text{ für n ungerade}$

$\Rightarrow$ Vorzeichenwechsel

"alterniernde" Folge; alterniernd = wechselnd

$\lim_{n\rightarrow\inf} [(-1)^n + \frac{1}{n}] = +1 \text{ für gerade;} -1 \text{ für ungerade}$

Die Folge $a_n = (-1)^n + \frac{1}{n}$ ist beschränk, da gilt:

$|a_n| \leq 1.5 = \frac{3}{2}$ | $c = 1.5$

$a_n \epsilon \ ]-1; 1.5]$

Grenzwerte von Folgen, deren Glieder aus Polynomen n-ten Grades bestehen:

\point{Beispiel}

$\lim_{n\rightarrow\inf} \frac{3n^2}{4n^2+n+5} \rightarrow \frac{3}{4}$ weil man mit $n^2$ alle Werte multipliziert

\point{Beispiel 2}

$\lim_{n\rightarrow\inf} \frac{3n^3}{4n^2 + n + 5} \rightarrow \inf$

\point{Beispiel 3}

$\lim_{n\rightarrow\inf} \frac{3n^2}{4n^3 + n + 5} = 0$

