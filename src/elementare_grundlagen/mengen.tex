
\subsection{Mengen}

\subsubsection{Menge der natürlichen Zahlen}
$\mathbb{N}$ = \{ 1, 2, 3, 4, ... \}

\point{Eigenschaften}
\begin{enumerate}
    \item Jede natürliche Zahl $\epsilon\mathbb{N}$ hat genau einen Nachfolger, nämlich $n + 1$
    \item Jede von 1 verschiedene natürliche Zahl $n$ hat genau einen Vorgänger $n-1$. Die Zahl 1 hat keinen Vorgänger.
\end{enumerate}

\point{Rechenoperationen}
\begin{itemize}
    \item $n_1 + n_2 \ \ \epsilon\mathbb{N}$
    \item $n_1 \cdot n_2 \ \ \epsilon\mathbb{N}$
\end{itemize}

\subsubsection{Menge der rationalen Zahlen}
$\mathbb{Q} = \{ \frac{a}{b} \ | \ a, b \ \ \epsilon\mathbb{Z} \text{ mit } b \neq 0 \}$

$ | \rightarrow$ mit der Eigenschaft


\point{Rechenoperationen}
\begin{itemize}
    \item +
    \item -
    \item \cdot
    \item /
\end{itemize}


\[
\frac{a_1}{b_1} = \frac{a_2}{b_2} <=> a_1 \cdot b_2 = a_2 \cdot b_1
\]

\[
\frac{a_1}{b_2} + \frac{a_2}{b_2} <=> \frac{a_1 \cdot b_2 + a_2 \cdot b_1}{b_1 \cdot b_2}
\]

\[
\frac{a_1}{b_1} \cdot \frac{a_2}{b_2} <=> \frac{a_1 \cdot a_2}{b_1 \cdot b_2}
\]


Rationale Zahlen lassen sich als periodische Dezimalzahlen darstellen: 

$\frac{6}{11} = 0.45454545 = 0.\overline{45}$


\subsubsection{Keine Rationale Zahlen}
\begin{itemize}
    \item \sqrt{2} = 1.141421
    \item \pi = 3.141592
\end{itemize}

$\Rightarrow$ Menge der reellen Zahlen $\mathbb{R}$

\point{Rechenoperationen}
\begin{itemize}
    \item +
    \item -
    \item \cdot
    \item /
\end{itemize}

$\mathbb{N} \subset \mathbb{Z} \subset \mathbb{Q} \subset \mathbb{R}$ 

$\rightarrow$ Alle $\mathbb{N}$ sind in $\mathbb{Z}$ enthalten, ...
