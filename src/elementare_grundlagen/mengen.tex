
\subsection{Mengen}

\subsubsection{Menge der natürlichen Zahlen}
$\mathbb{N}$ = \{ 1, 2, 3, 4, ... \}

\point{Eigenschaften}
\begin{enumerate}
    \item Jede natürliche Zahl $\epsilon\mathbb{N}$ hat genau einen Nachfolger, nämlich $n + 1$
    \item Jede von 1 verschiedene natürliche Zahl $n$ hat genau einen Vorgänger $n-1$. Die Zahl 1 hat keinen Vorgänger.
\end{enumerate}

\point{Rechenoperationen}
\begin{itemize}
    \item $n_1 + n_2 \ \ \epsilon\mathbb{N}$
    \item $n_1 \cdot n_2 \ \ \epsilon\mathbb{N}$
\end{itemize}

\subsubsection{Menge der rationalen Zahlen}
$\mathbb{Q} = \{ \frac{a}{b} \ | \ a, b \ \ \epsilon\mathbb{Z} \text{ mit } b \neq 0 \}$


