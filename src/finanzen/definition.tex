
\subsection{Definitionen}

\subsubsection{Einfache Verzinsung}
Keine Verzinsung der zinsen einer Periode in den nachfolgenden Perioden.

Wie viel erhält man für $800$\euro nach $2$ Jahren, wenn dieser Betrag mit $4,5\%$ p.a. (per anno)


$800$\euro: Kapital

$36$\euro: Zinsen (1. Jahr)

$36$\euro: Zinsen (2. Jahr)

$C_0$ Anfangskapital

$C_n$ Endkapital

$n$ Kaptialüberlassungsdauer

$i$ Zinssatz dezimal

$P$ Zinssatz/Zinsfuß in \%

\subsubsection{Allgemein - Einfache Verzinsung}
\[
    C_n = C_0 \cdot (1 + i \cdot n)
\]


\subsubsection{Zinseszinsliche Verzinsung}
Die Zinsen werden dem Kapital zugeschlagen und in Folgeperioden mitverzinst.

\point{Beispiel}

\[
    \begin{aligned}
        C_0 &= 800 \\
        i &= 0.045 \\
        n &= 2
    \end{aligned}
\]

$C_0$ = 800\euro
$C_1$ = $800 + 800 \cdot 0.045 = 836$\euro
$C_2$ = $836 + 836 \cdot 0.045 = 873.62$\euro

\subsubsection{Allgemein - Zinseszinsliche Verzinsung}
\[
    C_n = C_0 \cdot (1 + i)^n
\]

